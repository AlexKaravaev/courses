% указываем класс документа
\documentclass[12pt,a4paper,openany]{extarticle}

% подключаем собственный стилевой файл 
\usepackage{mystyle}

% указываем язык (для автоматической вставки слов, типа "Глава", "Содержание", "Литература", "рис." и пр.
\selectlanguage{russian}


\begin{document}

\newcounter{myappcounter}
\renewcommand{\themyappcounter}{\Alph{myappcounter}}
\newcommand{\myappnum}{\refstepcounter{myappcounter}\themyappcounter}

\part*{Заметка к видео с частью лекции}

\section{Доказательство того, что на поведение ДПТ при повороте его ротора П-регулятором состояния на желаемый угол $\theta_w$ оказывает влияние только вид матрицы $F$ (в частности ее характеристический полином)}
\hspace*{\parindent}Описывать состояние двигателя постоянного тока в указанной в заголовке ситуации можно как с помощью его вектора состояния $x = [\theta\;\omega\; I]^T$, так и с помощью ошибки $e$, определяемой из выражения:
\begin{equation}\label{eq:e_xw_x}
e = x_w - x = 
\begin{bmatrix}
\theta_w \\ 0 \\0
\end{bmatrix}
-
\begin{bmatrix}
\theta \\ \omega \\ I
\end{bmatrix}\!\!\ldotp
\end{equation}
Это обусловлено тем, что, например, если значение вектора $e$ будет известно в момент времени~$t_1$, то найти значение вектора $x$ в этот же момент времени можно обычной разностью:
\begin{equation}\label{eq:x_xw_e}
x = x_w-e\ldotp
\end{equation}

Подставим в модель вход-состояние-выход ДПТ:
\begin{equation}
\dot{x} = Ax+Bu,
\end{equation}
уравнение, описывающее работу П-регулятора состояния:
\begin{equation}
u = Ke\ldotp
\end{equation}
В итоге получим:
\begin{equation}\label{eq:dx_Ax_BKe}
\dot{x} = Ax+BKe\ldotp
\end{equation}

Продифференцируем выражение~\eqref{eq:e_xw_x} по времени, и с учетом того, что $x_w=const$, получим:
\begin{equation}
\dot{e} = \dot{x}_w - \dot{x} = -\dot{x}\ldotp
\end{equation}
Подставим в результат уравнение~\eqref{eq:dx_Ax_BKe}:
\begin{equation}
\dot{e} = -Ax-BKe\ldotp
\end{equation}
Добавим и вычтем к/из правой части последнего выражение $Ax_w$~--- в результате получим:
\begin{multline}\label{eq:de_Fe_Axw}
\dot{e} = Ax_w-Ax-BKe -Ax_w = A(x_w-x)-BKe-Ax_w =  Ae-BKe - Ax_w =\\= (A-BK)e - Ax_w = Fe - Ax_w\ldotp
\end{multline}

С~учетом того, что 
\begin{equation}
A=
\begin{bmatrix}
0 & 1 & 0\\
0 & 0 & \dfrac{k_m}{J}\\
0 & -\dfrac{ke}{L} & -\dfrac{R}{L}
\end{bmatrix}\!\!,
\end{equation} 
и содержания вектора $x_w$ выражение~\eqref{eq:de_Fe_Axw} может быть преобразовано к виду:
\begin{equation}
\dot{e} = Fe-
\begin{bmatrix}
0 & 1 & 0\\
0 & 0 & \dfrac{k_m}{J}\\
0 & -\dfrac{ke}{L} & -\dfrac{R}{L}
\end{bmatrix}
\begin{bmatrix}
\theta_w \\ 0 \\0
\end{bmatrix}
=Fe-
\begin{bmatrix}
0 \\ 0 \\ 0
\end{bmatrix}
=Fe,
\end{equation}
который наглядно позволяет увидеть тот факт, что поведение вектора $e$, а значит и $x$ (см.~\eqref{eq:x_xw_e}), определяется исключительно матрицей $F$.
\end{document}

