% указываем класс документа
\documentclass[12pt,a4paper,openany]{extarticle}
% подключаем собственный стилевой файл 
\usepackage{mystyle}
% указываем язык (для автоматической вставки слов, типа "Глава", "Содержание", "Литература", "рис." и пр.
\selectlanguage{russian}
\graphicspath{{./images/}}
\usepackage[pdftex]{lscape}

\begin{document}

\part*{Лабораторная работа №5\\
Робот с дифференциальным приводом}

\section{Методические рекомендации}
\hspace*{\parindent}До начала работы студент должен выполнить предыдущие лабораторные этого цикла.

\section{Теоретические сведения}
\hspace*{\parindent}В прошлой работе вы успели познакомиться c таким приемом управления, как ПИД-регулятор. Путем его реализации вам удалось реализовать процесс движения вдоль стены с минимальной ошибкой управления. В данной лабораторной работе будет предложено созать алгоритм движения широко используемого робота с дифференциальным приводом в заданную точку. Такой вид конструкции робота предполагает достаточно большую подвижность и мобильность вкупе со сравнительно легкой математической моделью. 


\paragraph*{Математическая модель робота}$\phantom{-}$\\
\hspace*{\parindent}
	
\end{document}
